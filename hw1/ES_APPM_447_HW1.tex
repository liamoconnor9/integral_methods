
\documentclass[10pt]{article}
\linespread{1.25}

%%Make Parenthesis scale to fit whats inside
\newcommand{\parry}[1]{\left( #1 \right)}

%% Language and font encodings
\usepackage[english]{babel}
\usepackage[utf8x]{inputenc}
\usepackage[T1]{fontenc}
\usepackage{subcaption}
\usepackage[section]{placeins}

%% Sets page size and margins
\usepackage[a4paper,top=3cm,bottom=2cm,left=3cm,right=3cm,marginparwidth=1.75cm]{geometry}

%% Useful packages
\usepackage{amsmath}
\usepackage{amssymb}
\usepackage{amsfonts}
\usepackage{mathtools}
\usepackage{graphicx}
\usepackage{xcolor}
\usepackage[colorinlistoftodos]{todonotes}
\usepackage[colorlinks=true, allcolors=blue]{hyperref}
\usepackage{enumerate}
\usepackage{enumitem}
\usepackage[framed,autolinebreaks,useliterate]{mcode} %% MATLAB code
\usepackage{siunitx}
\usepackage{float}
\usepackage{scrextend}
\usepackage[final]{pdfpages}

%%Header & Footer
\usepackage[myheadings]{fullpage}
\usepackage{fancyhdr}
\usepackage{lastpage}
\usepackage{graphicx, wrapfig, subcaption, setspace, booktabs}

%% Define \therefore command
\def\therefore{\boldsymbol{\text{ }
\leavevmode
\lower0.4ex\hbox{$\cdot$}
\kern-.5em\raise0.7ex\hbox{$\cdot$}
\kern-0.55em\lower0.4ex\hbox{$\cdot$}
\thinspace\text{ }}}
\renewcommand{\vec}[1]{\boldsymbol{#1}}

%% Units
\DeclareSIUnit\year{yr}
\DeclareSIUnit\dollar{\$}
\DeclareSIUnit\celcius{C^{\circ}}
\DeclareSIUnit\mole{mole}
\def\conclusion{\quad \Rightarrow \quad}

\begin{document}


%----------------------------------------------------------------------------------------
%	TITLE PAGE
%----------------------------------------------------------------------------------------

%----------------------------------------------------------------------------------------
% HEADER AND FOOTER
%----------------------------------------------------------------------------------------
\pagestyle{fancy}
\fancyhf{}
\setlength\headheight{12pt}
\fancyhead[L]{\textbf{ES-APPM 447: Boundary Integral Methods}}
\fancyhead[R]{\textbf{HW 1 \qquad 5/3/2021 \qquad Liam O'Connor}}
\fancyfoot[R]{Page \thepage\ of \pageref{LastPage}}

\section*{Problem Statement}

\begin{description}[wide = 0pt]

\item 1. Consider the Fredholm integral equation of the 2nd kind
\begin{equation}
    x(s) = \frac{3}{4} \cos (ks) + \frac{\sin (ks)}{4\pi k} [1 - (-1)^k] + \pi^{-1} \int_{0}^{\pi/2} \cos k(s + t) x(t) dt\label{eq:main}. 
\end{equation}
Note that the exact solution is given by $x_e(s) = \cos (ks)$. Check this. Set $k = 1$.

\begin{enumerate}[label=(\alph*)]
\item Solve this equation using the repeated Trapezoid rule. If $N$ is the total number of points, how large should it be to insure that the maximum error at each quadrature point is less than $10^{-6}$. Show this numerically. Suppose that you did not know the exact answer, how do you know when you have 6 decimal place accuracy in your answer? Show this numerically. Obtain the numerical convergence rate for your scheme.

\item Use a Gauss-Legendre quadrature method to solve the integral equation. Please clearly describe what you are doing. Obtain 6 decimal place accuracy. What is the total number of mesh points needed to get this accuracy? Suppose that you did not know the exact answer, how do you know when you have 6 decimal place accuracy in your answer? Show this numerically. Obtain the numerical convergence rate for your scheme.

\end{enumerate}


\item 2. Repeat parts (1.a) and (1.b) for $k = 4$ and $k = 15$. What are the differences between the three cases ($k = 1, 4, 15$). Why? If something no longer works, how should it be modified?

\section*{Classification and Context}
It is productive to rearrange (\ref{eq:main}) into the standard form of a Fredholm integral equation of the second kind. Recall that this standard form (given in the class notes) is 
\begin{align}
    x(s) &= y(s) + \lambda \int_a^b K(s, t) x(t) dt. \label{classify}
\end{align}
Letting $y(s) = \frac{3}{4} \cos (ks) + \frac{\sin (ks)}{4\pi k} [1 - (-1)^k]$, $\lambda = 1$, $K(s, t) = \pi^{-1}\cos k (s + t)$, $a = 0$, $b = \pi/2$, we see that these equations are indeed the same form. 

\subsection*{Exact Solution}
Letting $k = 1$, we have
\begin{align*}
    x(s) &= \frac{3}{4} \cos (s) + \frac{\sin (s)}{2\pi }  + \pi^{-1} \int_{0}^{\pi/2} \cos (s + t) x(t) dt
    \intertext{Letting $x(s) = \cos(s)$,}
    RHS &= \frac{3}{4} \cos (s) + \frac{\sin (s)}{2\pi }  + \pi^{-1} \int_{0}^{\pi/2} \cos (s + t) \cos(t) dt
    \intertext{Here we employ one of the ol' trig identities: recall $\cos (x) \cos (y) = \frac{1}{2}\big[ \cos (x - y) + \cos(x + y) \big]$. Therefore}
    &= \frac{3}{4} \cos (s) + \frac{\sin (s)}{2\pi }  + \frac{1}{2\pi} \int_{0}^{\pi/2} \cos (s) + \cos (2t + s) dt \\
    &= \frac{3}{4} \cos (s) + \frac{\sin (s)}{2\pi }  + \frac{1}{2\pi} \Big[ t\cos (s) + \frac{1}{2}\sin(2t + s) \Big]\Big|_{t = 0}^{t = \pi / 2}\\
    &= \frac{3}{4} \cos (s) + \frac{\sin (s)}{2\pi }  + \frac{1}{2\pi} \Big[ \frac{\pi}{2}\cos(s) + \frac{1}{2}\sin(s + \pi) - \frac{1}{2}\sin(s) \Big] \\
    &= \cos(s) = LHS.
\end{align*}
We demonstrate that the solution holds for all $k \in \mathbb{N}$ via induction. Suppose $\cos(ks)$ is indeed the general solution to (\ref{eq:main}). We seek to prove that $\cos ((k+1)s)$ is the general solution to (\ref{eq:main}) where we substitute $k+1$ for $k$ in the equation as well
\begin{align*}
    RHS &= \frac{3}{4} \cos ((k+1)s) + \frac{\sin ((k+1)s)}{2\pi (k+1)}  + \pi^{-1} \int_{0}^{\pi/2} \cos \big((k+1) (s + t)\big) \cos((k+1)t) dt
    \intertext{Using the same trig identity as above}
    &= \frac{3}{4} \cos ((k+1)s) + \frac{\sin ((k+1)s)}{2\pi }  + \frac{1}{2\pi} \int_{0}^{\pi/2} \cos ((k+1)s) + \cos \big((k+1)(2t + s)\big) dt \\
    &= \frac{3}{4} \cos ((k+1)s) + \frac{\sin ((k+1)s)}{2\pi }  \\
    &\quad + \frac{1}{2\pi} \Big[ \frac{\pi}{2}\cos ((k+1)s) + \frac{1}{2(k+1)} \sin \big((k+1)(\pi + s)\big) - \frac{1}{2(k+1)} \sin \big((k+1)(s)\big) \Big] \\
    &= \cos ((k+1)s) = LHS.
\end{align*}
Thus we have that $\cos (k s)$ is the solution for all positive integers $k$.

\section*{Numerical Quadrature}
Quadrature is used to approximate integrals numerically. This is important for studying integral-defined functions, solving integral equations, and performing numerical convolution. 
For this report we employ two strategies: repeated Trapezoid rule and Gauss-Legendre quadrature. 
With regard to this particular problem, we begin by discretizing the integral into the generalized summation form
\begin{align*}
    x(s) &= y(s) + \lambda \int_a^b K(s, \, t) x(t) dt \\
    &\approx y(s) + \lambda \sum_{j = 1}^N w_j K(s, \, t_j) x(t_j) dt 
\end{align*}
with $t_1$, $t_2$, ..., $t_N$ being the quadrature points. We anticipate regular values $\lambda = 1$ following the classification given by (\ref{classify}). We seek discrete approximations to the continuous function $x(t)$ at the quadrature points. To accomplish this, we define 
\begin{align*}
    \vec{x} &= [x_1, \, x_2, \, ..., \, x_N]^T = [x(s_1), \, x(s_2), \, ..., \, x(t_N)]^T \in \mathbb{R}^N\\
    \vec{y} &= [y_1, \, y_2, \, ..., \, y_N]^T = [y(s_1), \, y(s_2), \, ..., \, y(t_N)]^T  \in \mathbb{R}^N\\[0.3cm]
    K &= \begin{bmatrix}
        w_1K(s_1, t_1) & w_2K(s_1, t_2) & w_3K(s_1, t_3) & \dots  & w_N K(s_1, t_N) \\
        w_1K(s_2, t_1) & w_2K(s_2, t_2) & w_3K(s_2, t_3) & \dots  & w_N K(s_2, t_N) \\
        w_1K(s_3, t_1) & w_2K(s_3, t_2) & w_3K(s_3, t_3) & \dots  & w_N K(s_3, t_N) \\
        \vdots & \vdots & \vdots & \ddots & \vdots \\
        w_1K(s_N, t_1) & w_2K(s_N, t_2) & w_3K(s_N, t_3) & \dots  & w_NK(s_N, t_N)
    \end{bmatrix} \in \mathbb{R}^{N \times N}
\end{align*}
and rewrite (\ref{eq:main}) in matrix form
\begin{equation}
    \vec{x} = \vec{y} + K\vec{x} 
\end{equation}
thus we must solve the linear system
\begin{equation}
    \vec{x} (I - K) = \vec{y} \label{mateq}
\end{equation}
for $\vec{x}$. For simplicity we will use the default \texttt{Matlab} linear solver. The entries of $K$ depend on our choice of quadrature method.

\subsection*{Trapezoid Rule}
This is the simpler case. The Trapezoid rule is a type of Newton-Cotes method which involves the approximate tessellation of the region of integral with trapezoids. It follows arithmetically that the weights in this case are just equal to 1.0, except at the endpoints where they take on a value of 0.5.

\subsection*{Gauss-Legendre Quadrature}
This is the more complicated case. For Gaussian rules in general, we consider a truncated polynomial basis $\{q_n(s)\}$ whose elements are orthonormal on $(0,\, \pi/2)$. Each individual basis function of degree $k$ always has $k$ real roots (given in the Delves text). Here we choose the quadrature points points $t_j$ to be the roots of the $N$th order polynomial. Gauss-Legendre is the simplest version of this type of polynomial quadrature. If we take the weight function to be the multiplicative identity 1, then our polynomial basis ends up being the set of normalized, Legendre polynomials
\begin{align*}
    q_i(s) &= \sqrt{\frac{2i + 1}{2}} P_i(s), \quad i = 0, 1, ..., N-1 \quad \quad \text{(Delves pg. 30)} 
    \intertext{the appropriate weights$^1$ are then given by}
    w_i &= \frac{2}{(1 - t^2_i)[ P'_n(t_i) ]^2}, \quad i = 0, 1, ..., N-1
\end{align*}
where $P_i(s)$ is the Legendre polynomial of order $i$. Legendre polynomials form a complete basis on the interval $(-1, 1)$. Accordingly, we must perform a linear transformation$^1$ to the general interval $(a, b)$
\begin{align*}
    \int_a^b f(x) dx &= \frac{b - a}{2} \int_{-1}^1 f\Big( \frac{b - a}{2}\xi + \frac{a + b}{2} \Big) \frac{dx}{d\xi} d\xi \\
    &\approx \Big(\frac{b - a}{2}\Big)^2 \sum_{j = 1}^{N} w_i f\Big( \frac{b - a}{2}\xi + \frac{a + b}{2} \Big).
\end{align*}
With this in mind we direct our focus towards numerical results. 

$^1$Gauss-Legendre weights and generalized interval transform are given on the handy-dandy Wikipedia page: \url{https://en.wikipedia.org/wiki/Gaussian_quadrature}.

    % \begin{figure}[H]
    %     \centering
    %     \includegraphics[width=2in]{XXXX.PNG}
    %     \caption{Maybe we need a figure}
    %     \label{XXX}
    % \end{figure}
    
    % \begin{figure}[H]
    %     \centering
    %     \subfloat{\includegraphics[height=4.99cm]{XXX.PNG}}
    %     \hfill
    %     \subfloat{\includegraphics[height=4.99cm]{XXX.PNG}}
    %     \caption{maybe we need more than one figure} 
    %     \label{alpha}
    % \end{figure}

\end{description}

\section*{Results}

\section*{Code}

\begin{lstlisting}
%%%%%%%%%%%%%%%%%%%%%%%%
%% QUESTION C
%%%%%%%%%%%%%%%%%%%%%%%%

alpha = 0.1;
beta = 0.1;
a = 0.25;
b = 1.0;
\end{lstlisting}
\end{document}
